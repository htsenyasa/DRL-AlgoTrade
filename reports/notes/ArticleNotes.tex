\documentclass[twocolumn,aps,pra,superscriptaddress,nofootinbib,longbibliography]{revtex4-2}
\usepackage{graphicx}% Include figure files
\usepackage{dcolumn}% Align table columns on decimal point
\usepackage{bm}% bold math
\usepackage[latin5]{inputenc}% For Turkish characters
\usepackage{amssymb}
\usepackage{amsmath}
\usepackage{amsthm}
\usepackage{amsbsy}
%\usepackage{mathptmx}
\usepackage{amsfonts}
\usepackage{mathtools}
\usepackage{physics}
\DeclareMathAlphabet{\mathcal}{OMS}{cmsy}{m}{n}
\usepackage{tabularx}
\usepackage{xcolor}
\usepackage{mathtools}
\usepackage{oubraces}
\usepackage{dsfont}
\usepackage[title, titletoc]{appendix}
\usepackage{graphicx}
\usepackage{rotating}
\usepackage[most]{tcolorbox}
\usepackage{enumerate}
\usepackage{subfloat}
%\usepackage{hyperref}% add hypertext capabilities
\usepackage[colorlinks=true,linkcolor=blue,citecolor=blue,urlcolor=blue]{hyperref}%
\usepackage{cleveref}
\usepackage[normalem]{ulem}
\Crefname{subfigures}{figure}{figures}
\Crefname{subfigures}{Figure}{Figures}


\newcommand{\tcr}[1]{{\color{red} #1}}
\newcommand{\tcb}[1]{{\color{blue} #1}}
\newcommand{\tcg}[1]{{\color{gray} #1}}
\newcommand{\vari}[1]{\text{Var}(#1)}


\def\bra#1{\mathinner{\langle{#1}|}}
\def\ket#1{\mathinner{|{#1}\rangle}}
%\def\braket#1{\langle{#1}|\rangle}
\def\Bra#1{\left<#1\right|}
\def\Ket#1{\left|#1\right>}
%\newcommand{\dd}{\differential}

%\newcommand{\nc}{\newcommand}
%\nc{\rnc}{\renewcommand}
%\nc{\beg}{\begin{equation}}
%\nc{\eeq}{{\end{equation}}}
%\nc{\beqa}{\begin{eqnarray}}
%\nc{\eeqa}{\end{eqnarray}}
%\nc{\lbar}[1]{\overline{#1}}
%\nc{\bra}[1]{\langle#1|}
%\nc{\ket}[1]{|#1\rangle}
%\nc{\ketbra}[2]{|#1\rangle\!\langle#2|}
%\nc{\braket}[2]{\langle#1|#2\rangle}
%\newcommand{\braandket}[3]{\langle #1|#2|#3\rangle}


\newtheorem{theorem}{Theorem}
\newtheorem{lemma}[theorem]{Lemma}
\newtheorem{definition}[theorem]{Definition}
\newtheorem{corollary}[theorem]{Corollary}
\newtheorem{observation}[theorem]{Observation}
\newtheorem{proposition}[theorem]{Proposition}
\newenvironment{proof1}{\noindent\textbf{Proof:}}{\hfill \( \blacksquare \) \newline}



%***=============================================================***%
%***=============================================================***%


\begin{document}

\title{Deep Learning Project Progress Report}

\author{H\"useyin Talha \c{S}enya\c{s}a}
\email{senyasa@itu.edu.tr}
\affiliation{Department of Physics, Faculty of Science and Letters, Istanbul Technical University, 34469 Maslak, Istanbul, Turkey}




%\date{\today}

%***=============================================================***%
%***=============================================================***%

\begin{abstract}
My personal notes on \textit{An application of deep reinforcement learning to algorithmic trading}.

\end{abstract}

\maketitle

%***=============================================================***%
%***=============================================================***%




\section{Implementation of the Framework}

Some definitions and symbols

\begin{enumerate}
    \item \(S\): Set of environment and agent state.
    \item \(A\): Set of actions which are available for agent to use.
    \item \(s_t\): RL environment internal state
    \item \(o_t\): Observation
    \item \(a_t\): Trading action 
    \item \(i_t\): Information
    \item \(\pi(a_t|i_t)\): Trading policy (Rule)
    \item \(r_t\): Network's reward.
    \item \(\nu_{t}^c\): Total amount of cash in portfolio. 
    \item \(\nu_{t}^s\): Corresponding value of the share.
    \item \(n_t\): Total number of shares, lots. 
\end{enumerate}

Reinforcement learning techniques are concerned with the design of \(\pi\) maximizing an optimality criterion, which directly depends on the immediate rewards \(r_t\) observed over a certain time horizon.  

\subsection{Trading Environment}

\subsubsection{Implementation of Fundamental Operations}

Buy, Sell, GoShort, GoLong

\subsubsection{Implementation of Trading Environment via OpenAI Gym}

The trading environment of the DQL is implemented in \textit{OpenAI Gym} framework. The elements of the trading environment are as follows:

\begin{equation}
    \begin{aligned}
        \mathcal{E}_{TE} = \{&\text{Close, Low, High, Volume, Position,} \\
        &\text{Action, Holdings, Cash, Money, Returns}\}
    \end{aligned}
\end{equation}

\subsection{Implementation of Deep Neural Network}

Network Architecture, loss function etc.

\subsection{Implementation of Double-Q Mechanism}

\section{Additional notes and questions}

\begin{enumerate}
    \item Representing the transition from one candle to the next one as a Markov process.
    \item Considering the correlation between candles as a spin system (as in the case of Witten's ``An introduction to quantum information theory'') 
\end{enumerate}




\clearpage


\bibliography{biblio}

\end{document}
